\documentclass[11p, a4paper]{article}
\usepackage{fkssugar}
\textwidth 16cm \textheight 24.6cm
\topmargin -1.3cm
\oddsidemargin 0cm
\usepackage[czech]{babel}
\usepackage[utf8]{inputenc}
\usepackage{wrapfig}
\usepackage{graphicx}
\begin{document}
Pokud použiju rovnici pohybu pro "spojitý" žebřík padající z výšky
\eq{
    \ddot{x}=g+\frac{\dot{x}^2}{L-x}\frac{I}{p_0}
}
\begin{wrapfigure}{r}{0.5\textwidth}
      \vspace{-20pt}
      \begin{center}
      \includegraphics[width=0.48\textwidth]{../src/z1.mps}
      \end{center}
      \vspace{-20pt}
\end{wrapfigure}
Klíčové je pro zjistit jak velký impuls bude předán od dopadajícího článku ke
zbytku žebříku. Při tom použiju prozatím použiju následující aproximace první
z ních je, že úhel, který svírá šprična s vodorovnou plochou $\alpha\approx0$.
Další z nich je zanedbání hmotnosti jednoho článku $m<<M$ vůči hmotnosti zbytku
žebříku $M$. Další je dokonalá tuhost článku. Posledně je celý příklad počítán
s tím, že článek se se zemí srazí dokonale nepružně. Prvně si zavedu důležité
pojmy na obrázku 1.
\eq[m]{
    l=2|\mathrm{DS}|\,,d=2|\mathrm{DT}|\lambda=\frac{m}{l}\\
    L_D=\frac{1}{2}mvl\,,J_D=\frac{1}{3}l^2m\,,L_D=J_D\cdot\omega_0=\frac{1}{2}mvl\\
    \omega_0=\frac{3}{2}\frac{v}{l}\ztoho v_T=d\omega_0=\frac{3}{2}\frac{d}{l}v
}

\eq[m]{
    \omega=\frac{v}{d}\,,Id=\Delta \omega J_D=J_D(\omega_0-\omega)=
    \frac{3d-2l}{6d}lmv\\
    I=\frac{l}{d}\frac{3d-2l}{6d}mv
}
\end{document}
